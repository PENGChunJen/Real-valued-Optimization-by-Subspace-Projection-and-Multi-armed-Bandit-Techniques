\chapter{The New Bandit Technique}
\label{chapter:new_bandit}

In this chapter, we illustrate the proposed algorithm step-by-step. 

Overview of real-valued optimization

\section{Framework of the New Bandit Algorithm}

The goal is to identify the ROI for exploitation and maintain exploration through resource allocation.

Each algorithm needs to be modified to satisfy the following conditions:
\begin{enumerate}
    \item Update one individual at a time 
    \item The algorithm can be projected onto a subspace and continue iterating
    \item Replace one individual with a given position and fitness
\end{enumerate}

The pseudocode of our new algorithm is given in Algorithm~\ref{algo:new_bandit}

\begin{algorithm}%[t!]
\caption{Framework of the new Bandit Algorithm}\label{algo:new_banidt}

$\boldsymbol{X}_{n}$: solution of the $n^{th}$ iteration, $\sigma_n$: step size of the $n^{th}$ iteration, \\
$N(\boldsymbol{0}, \boldsymbol{I})$: multivariant normal distribution with mean vector $\boldsymbol{0}$ \\ 
and identical covariance matrix $\boldsymbol{I}$.

\BlankLine
\SetKwInOut{Input}{input} \SetKwInOut{Output}{output}
\Input{ $f$: evaluation function }
\Output{ $X_{n+1}$: best solution }

\BlankLine
Initialize $\boldsymbol{X}_0, \sigma_0$ \\
\While{ termination criterion not met } {

    $\widetilde{\boldsymbol{X}}_n = \boldsymbol{X}_n + \sigma_n N(\boldsymbol{0}, \boldsymbol{I})$  \\

    \eIf{ $f(\widetilde{\boldsymbol{X}}_n) \leq f(\boldsymbol{X}_n) $}{
        $\boldsymbol{X}_{n+1} = \widetilde{\boldsymbol{X}}_n$ \\
        $\sigma_{n+1} = 1.5 \sigma_n$
    }{
        $\boldsymbol{X}_{n+1} = \boldsymbol{X}_n$ \\
        $\sigma_{n+1} = 1.5^{-1/4}\sigma_n$
    }
}

\Return $\boldsymbol{X}_{n+1}$

\end{algorithm}





\section{Initialization and Unimodal Identification}

For a problem with $D$ dimension, initialize with $100D$ points.

Keep only top $50\%$ of points for unimodal identification. 
Use clustering techniques mentioned in Chapter~\ref{chapter:clustering}.

Iteratively add $10D$ points until estimated cluster number is identical.

With a given cluster number $k$, do K-Means clustering.





\section{Define Region of Interest}

Each arm is composed of an algorithm and a projection matrix.
Initial matrix is set as a tight hyperbox that contains all points in the given cluster.
Optimize matrix one-by-one.
Resize each cluster to match the required population for each algorithm.
Start algorithm with given initial positions and fitnesses.





\section{Remain Evaluations Allocation}

Calcualte remain evaluations allocatoin and normalize the value to 0~1.
Add newly calculated allocatoin to record.
Choose arm the argmax allocation to pull, i.e. update one individual.
Max arm allocation -1.





\section{Recluster}



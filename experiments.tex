\chapter{Experiments}
\label{c:experiments}

% talk about ROI, what it is, why it is important
Overview of real-valued optimization

\section{Test Problems}

\subsection{CEC2005 25 benchmark problems}
Describe termination criterion and evaluation method for CEC2005
Describe reason for 25 repeat runs.
Describe how initial setting effect real-valued optimization, so that all algorithms should start with identicle initial status.


\section{Experiment Settings}

Describe the parameters setting for CMA-ES, SPSO and ACOR
CMA-ES initial mean and std, population settings
SPSO2011 parameters (c, w) settings, and population settings
For ACOR, we set our parameters according to the original paper~\ref{Socha:2008:ACOR}. 
The parameters are shown in Table~\ref{table:ACOR_parameters}.


\begin{table}%[t!]
\centering
\label{table:ACOR_parameters}
\begin{tabular}{lll}
\hline
Parameter                        & Symbol   & Value          \\ \hline
No. of ants used in an iteration & $m$      & $2$            \\
Speed of convergence             & $\xi$    & $0.85$         \\
Locality of the search process   & $q$      & $10^{-4}$      \\
Archive size                     & $k$      & $50$           \\ \hline
\end{tabular}
\caption{Summary of the parameters used by $ACO_R$}
\end{table}


Describe our bandit parameters setting, including the initial population, maximum number of arms, and (1+1)-ES step size.
